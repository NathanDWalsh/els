\chapter{Project}\label{project}

This project has three main modules that will be developed in concert:
eel-project, a folder-based project specification with eel-yaml, a
pipeline configuration language, and eel-cli, a command line tool which
interprets and executes eel-project. Figure \ref{fig:sequence} below is
a high-level sequence diagram as to how these modules interact with a
data pipeline.

\includesvg{img/sequence.svg}

\chapter{Principal Goals}\label{principal-goals}

This scope of this project will focus on the usability aspect of the
language and tool:

\begin{itemize}
\item
  Design a declarative language (YAML) and accompanying project
  structure for extracting and loading data to/from standard file
  formats or databases, including in-process memory structures as
  targets (i.e., pandas data frames).
\item
  Design CLI system to interpret language and optimally orchestrate
  extract/loading procedures, allowing for configuration trees (for
  inheriting config from parent branches) and inferred configuration
  (i.e., type inference, reading existing meta-data from sources).
\item
  Allow non-contextual transformations to be defined (i.e.~column which
  defines source, index column)
\end{itemize}

\chapter{Secondary Goals}\label{secondary-goals}

These will not be built into the initial project, but design
considerations will be taken in order to allow for the implementation of
the following features.

\begin{itemize}
\item
  Column-level data providence should be built into the system.
\item
  Ability to optimize extract-load procedures based on available tools
  in the executing system, for example utilizing database-native
  extract-load procedures when using a particular database as a
  source/target.
\item
  Multi-thread and/or multi-process when speed/performance gains likely.
\item
  Contextual transformation support.
\end{itemize}

\chapter{Background}\label{background}

a short description of how previous work addresses (or fails to address)
this problem.

\chapter{Methods}\label{methods}

A description of the methods and techniques to be used, indicating that
alternatives have been considered and ruled out on sound scientific or
engineering grounds.

\chapter{Evaluation}\label{evaluation}

Details of the metrics or other methods by which the outcomes will be
evaluated.

\chapter{legal, social, ethical or professional
issues}\label{legal-social-ethical-or-professional-issues}

\chapter{Workplan}\label{workplan}

A timetable detailing what will be done to complete the proposed
project, and when these tasks will be completed.

\section{eel-yaml}\label{eel-yaml}

A human-readable declarative configuration language defined in a YAML
schema that can be used in most popular editors (such as Visual Studio
Code) to facilitate the user's ability to create the extract-load
configurations manually. The declarative language can be considered a
configuration which details data targets and sources.

\section{eel-project}\label{eel-project}

A folder/directory representing the top level of an eel project, all
subfolders and files are considered part of the project's contents. All
folders and certain file types are handled as follows:

\begin{itemize}
\item
  Subfolders are used to organize a hierarchy of targets and/or sources
  in the pipeline, a special configuration file \_.eel.yml is used to
  set the configuration that will be inherited by all containing files
  and folders.
\item
  Recognized data files such as .csv and .xlsx will be treated as
  implicit source configurations, so a .csv file will be treated as
  first a configuration file which refers to itself as the source file.
\item
  Complimentary eel.yml files can augment implicit source configurations
  with additional information such as encoding of the csv. In for a
  complimentary file to be recognized as such, it should have the exact
  same name as the recognized data file with the additional .eel.yml
  extension. For example: data1.csv is recognized as an implicit source
  while data1.csv.eel.yml explicitly defines the encoding that should be
  used.
\item
  Sole eel.yml files (those which are not paired to a data file) may
  contain data source and/or target configurations.
\end{itemize}

The project will stress DRY principals allow for minimum configurations
using some of the following methods:

\begin{itemize}
\item
  Configuration inheritance via directory/folder structure:
  configurations inherit from parent node (i.e., folder). Useful when a
  target object (i.e.~database or table) is the same for multiple data
  sources.
\item
  Use existing metadata (i.e., from database schema) for data types.
\item
  Type inferencing when no data type information available, for example
  in csv files.
\end{itemize}

Configuration and language are two terms that are used interchangeably
in this document, and both refer broadly to the eel-yaml and eel-project
modules.

\section{eel-cli}\label{eel-cli}

A minimum viable product for a command line tool that will interpret
eel-project and perform certain actions:

\begin{itemize}
\item
  Show a preview of how eel-cli interprets the current eel-yaml project:
\item
  config tree
\item
  task flow tree
\item
  eel-yaml inherits
\item
  execute dataflow
\end{itemize}

execute the extract-load procedures and also provide granular
information on the extractions.

\chapter{Roadmap}\label{roadmap}

Although these three parts of the project are considered complimentary
and will be developed in concert, it is expected that they would
eventually split and evolve into separate projects in order to separate
the declarative language from any future interpreters that could be
developed to support it.

\chapter{Examples for Citations}\label{examples-for-citations}

Here are examples for citations \citep{P2} and \citep{P2}.

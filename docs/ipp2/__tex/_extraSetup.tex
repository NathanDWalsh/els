% Use this to include the class that the document loads and any document classes further in the stack
\PassOptionsToClass{
  11pt
  %a4paper
}                      {article}
\PassOptionsToClass{
  11pt
  %a4paper
}                      {report}


% You also need to pass the page size to geometry because \pdfpagewidth and \pdfpageheight are undefined by default in lualatex
\PassOptionsToPackage{
   export% export options to graphicx package for the \includegraphics command
}                      {adjustbox}
\PassOptionsToPackage{
   tbtags% align equation numbers to the top or bottom of a multi-line equation
}                      {amsmath}% loaded by mathtools
\PassOptionsToPackage{
   backend=biber% biber is 21st century LaTeX: 20th century LaTeX used bibtex
   % style is set in the main document
  %,style=bath,sorting=nyt% harvard referencing sorted by author name then year then title
  % ,style=ieee% numeric citations
  ,uniquename=false
  ,uniquelist=false
  ,giveninits=false
  ,indexing=true% allow building indices globally; "cite" for citations only, "bib" for bibliography only
  ,doi=true
  ,url=true
  %,backref% remove this if you don't want the references list to contain hyperlinks to where the citations were made
  ,backrefstyle=none% the default is three: this changes backreferencing page compression for consecutive pages
  ,backrefsetstyle=setonly
  ,backreffloats=true
  %,backrefpage={Cited on page~}% not working on TeXLive23, should work from TeXLive24
  %,backrefpages={Cited on pages~}% not working on TeXLive23, should work from TeXLive24
}                      {biblatex}
\PassOptionsToPackage{
   font={normalcolor, small, stretch=1.4}% stretch is the caption line spacing
  ,hypcap=true% true = hyperlink to beginning of table/figure; false=hyperlink to caption
  ,hypcapspace=\baselineskip% amount of page to show above the table/figure when hyperlinking
  ,labelfont={bf, normalcolor}% bf is bold, sc is small caps, sf is sans, rm is serif
  ,listformat=simple% how the caption appears in the List of Tables/Figures
  ,labelformat=simple% how the caption is formmatted in the text
  ,labelsep=space% colon
  ,format=hang% use plain if you don't want a hanging indent
  ,singlelinecheck=true% false stops a one-line caption being centred
  ,tableposition=below% assumes captions are underneath tables
  ,textformat=period% automatically add a . to the end of the caption text
  ,width=.75\textwidth%
  %,aboveskip=10pt% change the space above the caption
  %,belowskip=10pt% change the space below the caption
}                      {caption}
\PassOptionsToPackage{}{subcaption}
\PassOptionsToPackage{
    capitalise
   ,nameinlink
   ,noabbrev
   ,sort&compress
 }                     {cleveref}
\PassOptionsToPackage{
    en-GB
   ,useregional=false
   ,showseconds=false
}                      {datetime2}
 \PassOptionsToPackage{
   shortcuts% provides \=/ for non-breaking hyphens and \-/ to allow a word to be hyphenated before the first hyphen (normally LaTeX doesn't hyphenate the first part)
}                      {extdash}
\PassOptionsToPackage{
   quiet
}                      {fontspec}
\PassOptionsToPackage{
   hang
}                      {footmisc}
% You also need to pass the page size to geometry because \pdfpagewidth and \pdfpageheight are undefined by default in lualatex
\PassOptionsToPackage{
   a4paper
}                      {geometry}
\PassOptionsToPackage{
    numberedsection=false
   ,nonumberlist
   %,section=section% Uncomment if you always the Glossary to be a section, otherwise the code below uses chapter if available otherwise section
   ,toc=true
}                      {glossaries}
% This code uses section=chapter if chapter is defined by the documentclass, otherwise section=section
% If you always want to use section=section regardless, then uncomment section=section above and comment out the line below
\makeatletter\@ifundefined{chapter}{\PassOptionsToPackage{section=section}{glossaries}}{\PassOptionsToPackage{section=chapter}{glossaries}}\makeatother


\PassOptionsToPackage{
   abbreviations
  ,symbols
  ,xindy
}                      {glossaries-extra}
\PassOptionsToPackage{
   %pdftex% only if compiling with pdftex
}                      {graphicx}
\PassOptionsToPackage{
   pdfa
}                      {hyperref}
% idxlayout is a convenient and reliable way to specify default layouts for indices. You can override these settings for a specific index
\PassOptionsToPackage{
   columns=3
  ,indentunit=\parindent
  ,columnsep=1.5\parindent
  ,totoc,
  font=small
}                      {idxlayout}
\PassOptionsToPackage{
   xindy% 21st century LaTeX uses xindy + .xdy config file; 20th century LaTeX uses makeindex+ 
  ,nonewpage% comment this out if you want the index to start a new page, eg if you are using the report of book class
  ,original
  ,noautomatic% LEAVE THIS ONE especially on Overleaf otherwise you run the risk of the index being processed again
}                      {imakeidx}
\PassOptionsToPackage{
   utf8
}                      {inputenc}
\PassOptionsToPackage{
  ,contentBlocks
  ,debugExtensions
  ,definitionLists
  ,fancy_lists
  ,fencedCode
  ,hashEnumerators
  ,inlineNotes
  ,jekyllData
  ,lineBlocks
  ,notes
  ,pipeTables
  ,rawAttribute
  ,smartEllipses
  ,strikeThrough
  ,subscripts
  ,superscripts
  ,tableCaptions
  ,taskLists
  ,texMathDollars
  ,texMathDoubleBackslash
  ,texMathSingleBackslash
}                      {markdown}
\PassOptionsToPackage{
   protrusion
  ,stretch=20
  ,shrink=20
}                      {microtype}
\PassOptionsToPackage{
   l2tabu
  ,orthodox
}                      {nag}% One function of nag is tell you how old-fashioned your LaTeX is
\PassOptionsToPackage{
   footnote=true% Sidenotes instead of footnotes
}                      {snotez}% Sidenotes
\PassOptionsToPackage{
   %explicit% this means you must use #1 to print the name in titleformat commands otherwise compilation hangs
  ,pagestyles% automatically load the page styles package titleps
}                      {titlesec}% this is instead of the fancyhdr package and must be loaded before hyperref
\PassOptionsToPackage{
   titles% lets this package play better with other packages
}                      {tocloft}% tools for putting Lists of Figures and Tables on the contents page
\PassOptionsToPackage{
   normalem% needed if the sectsty package is loaded (which for Informatics reports it is)
}                      {ulem}% use UnderLining instead of EMphasis (italics)
\PassOptionsToPackage{
   svgnames
  ,table
  ,usenames
}                      {xcolor}% xcolor is loaded by some common packages

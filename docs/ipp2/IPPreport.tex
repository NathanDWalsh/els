% Instructions for editing this file are in readme.md

%%%%%%%%%%
% Document meta data
\newcommand*{\AUTHOR}{Student Name}
\newcommand*{\TITLE}{Project Title\\ Possibly Across Two Lines}
\newcommand*{\SUPERVISOR}{Prof~S Visor}
\newcommand*{\IPPTUTOR}{Dr~IPP Tutor}
\newcommand*{\KEYWORDS}{}% comma-separated list

%%%%%%%%%%
% Abstract
% Edit the abstract in abstract.tex


%%%%%%%%%%
% Colours
% Edit colours.tex if you want to change the colours of the
% text, page, and School logo
% Dark themes, and dyslexia-friendly themes are available
% Check with your IPP tutor and your project supervisor what colour scheme they prefer


%%%%%%%%%%
% Referencing
% Use ONE of the following to set the bibliography and referencing styles:
\PassOptionsToPackage{style=bath,sorting=nyt}{biblatex}% Harvard-style, sorted by author name then year then title
% \PassOptionsToPackage{style=ieee}{biblatex} % Numeric-style

% Comment out the following line if you do NOT want the references list to contain hyperlinks back to where they are cited in the main text
\PassOptionsToPackage{backref}{biblatex}


%%%%%%%%%%
% The document class: leave this alone   :)
% To base font size can be 10pt, 11, or 12pt
\documentclass[11pt]{ipp}


%%%%%%%%%%
% Font
% Changes to the font and line spacing do not affect the title page

% Open Dyslexic font
% \setmainfont{Open Dyslexic} % replaces the main serif font (body text, notes, and captions)
% \setsansfont{Open Dyslexic} % replaces the main sans font (headings)
% \setmonofont{Open Dyslexic} % replaces monospaced font (URLs and program code)

% General change to the font: uncomment these lines and replace the font names
% \setmainfont{Merriweather Light} % replaces the main serif font (body text, notes, and captions)
% \setsansfont{Merriweather Sans} % replaces the main sans font (headings)
% \setmonofont{Anonymous Pro} % replaces monospaced font (URLs and program code)


% You may change the url font
\renewcommand\UrlFont{\ttfamily\footnotesize} %use the specified mono font but smaller

% Set the sequence of styles used in nested \emph{} commands: you may change this
\emfontdeclare{\bfseries,\itshape,\scshape}


%%%%%%%%%%
% Line spacing
% This affects document line spacing (not the tables or the tables of contents)
\setstretch{1.3}

% The line spacing of tables: you may change this
\renewcommand{\arraystretch}{1.5}

% Lets captions be wider; stretch= changes caption's line spacing
% \captionsetup{width=.8\textwidth,{normalcolor, small, stretch=1.3}}

% The line spacing in the bibliography
\AtBeginBibliography{\normalfont\small\RaggedRight\setstretch{1.2}}



%%%%%%%%%%
% Line justification
% To stop text being fully justified then use ONE of the following:
% EITHER allow a small amount of hypenation at the ends of lines
% \setlength{\RaggedRightParindent}{\parindent}\RaggedRight
% OR allow no hyphenation at the ends of lines
% \newlength{\originalParindent}\setlength{\originalParindent}{\parindent}\raggedright\setlength{\parindent}{\originalParindent}


%%%%%%%%%%
% Packages
% DON'T ADD PACKAGES HERE: put them in _extraPackages.tex
% but first check if they are already loaded,
% paying attention to each package's manual to see whether it must be loaded
% before biblatex, before hyperref, or after hyperref

% Uncomment if you want syntax diagrams
% \usepackage{syntaxdi}%

% If you want pseudocode, consider one of these packages, all updated in 2023:
% https://ctan.org/pkg/pseudo
% https://ctan.org/pkg/algpseudocodex
% https://ctan.org/pkg/algxpar



%%%%%%%%%%
% Bibliography file
% Only one bibliography file can be specified
\addbibresource{IPPrefs.bib}


%%%%%%%%%%
% Hyphenation of words
\pghyphenation[variant=british]{english}{algor-ithm algor-ithms book-ends inter-net LaTeX mono-spaced}


%%%%%%%%%%
% Warning messages
% If you wish to see warning messages about sidenotes moving, then comment the next line out
\WarningFilter*{latex}{Marginpar on page \thepage\space moved}


%%%%%%%%%%%%%%%%%%%%%%%%%%%%%%
\begin{document}
% In case you're wondering, the % signs ending every line on the titlepage are to suppress all whitespace and unintentional ending of paragarphs
\bookmarksetup{startatroot}\bookmark[named=FirstPage, level=0]{Title page}\bookmarksetup{startatroot}%

\newcommand*{\IPPTITLE}{\normalfont\titlepageFontSans\LARGE{}Informatics Project Proposal~\the\year}%
\newcommand*{\TITLEPAGETITLE}{\normalfont\titlepageFontSans\Huge\bfseries\TITLE}%
\newcommand*{\TITLEPAGEAUTHOR}{\normalfont\titlepageFontSerif\huge\AUTHOR}%
\newcommand*{\SCHOOL}{School of Informatics}%
\newcommand*{\INSTITUTION}{University of Edinburgh}%
\newcommand*{\SUPERVISORROLE}{Supervisor}%
\newcommand*{\IPPTUTORROLE}{IPP~Tutor}%


\newsavebox{\ippbox}%
\savebox{\ippbox}{%
  \begin{varwidth}{\linewidth}%
    \centering\IPPTITLE%
  \end{varwidth}%
}%

\newsavebox{\titlebox}%
\savebox{\titlebox}{%
  \begin{varwidth}{\linewidth}%
    \centering\TITLEPAGETITLE%
  \end{varwidth}%
}%

\newsavebox{\authorsbox}%
\savebox{\authorsbox}{%
  \begin{varwidth}{\linewidth}%
    \centering\TITLEPAGEAUTHOR%
  \end{varwidth}%
}%

% \ifundef{\PROJECTTYPE}%
%   {}
%   {%
%     \newsavebox{\projecttypebox}%
%     \savebox{\projecttypebox}{%
%       \begin{varwidth}{\linewidth}%
%         \begin{singlespace}%
%           \centering\LARGE\titlepageFontSerif\addfontfeatures{Numbers={Proportional,Lining}}\PROJECTTYPE%
%         \end{singlespace}%
%       \end{varwidth}%
%     }%
%   }%

% \ifundef{\PROJECTNOTE}%
%   {}%
%   {%
%     \newsavebox{\projectnotebox}%
%     \savebox{\projectnotebox}{%
%       \begin{varwidth}{\linewidth}%
%         \begin{singlespace}%
%           \centering\normalfont\LARGE\bfseries\titlepageFontSerif\addfontfeatures{Numbers={Proportional,Lining}}\PROJECTNOTE%
%         \end{singlespace}%
%       \end{varwidth}%
%     }%
%   }%

\newsavebox{\abstractbox}%
\savebox{\abstractbox}{%
  \begin{varwidth}{.8\linewidth}%
    \normalfont\titlepageFontSerif\setstretch{1.2}\addfontfeatures{Numbers={Proportional,Lining}}TBD
%
  \end{varwidth}%
}%

\newsavebox{\staffbox}%
\savebox{\staffbox}{%
  \normalfont\titlepageFontSerif\Large%
  \begin{varwidth}[t]{0.45\linewidth}%
    \begin{flushleft}%
        \begin{singlespace}\SUPERVISORROLE\par\bfseries\SUPERVISOR\par\end{singlespace}%
    \end{flushleft}%
  \end{varwidth}%
  \hspace*{6em}%
  \begin{varwidth}[t]{0.45\linewidth}%
    \begin{flushright}%
      \begin{singlespace}\IPPTUTORROLE\par\bfseries\IPPTUTOR\par\end{singlespace}%
    \end{flushright}%
  \end{varwidth}%
}%

\newcommand*{\titleRuleWidth}{0.2ex}%
\newcommand*{\titleRule}{\rule{\wd\titlebox}{\titleRuleWidth}}%
\newcommand*{\ippRule}{\rule{\wd\ippbox}{\titleRuleWidth}}%
\newcommand*{\authorsRule}{\rule{\wd\authorsbox}{\titleRuleWidth}}%

\newgeometry{margin=1cm}%
  \begin{singlespace}%
    \centering%
    \vspace*{\fill}%
    \titleRule\\[1.5\baselineskip]%
    \thispagestyle{empty}%
    \usebox{\ippbox}\par%
    \vspace*{1\baselineskip}%
    \ippRule\\[1\baselineskip]%
    \vspace*{1\baselineskip}%
    \usebox{\titlebox}\par%
    \vspace*{1.5\baselineskip}%
    \authorsRule\\[1.4\baselineskip]%
    \usebox{\authorsbox}\par%
    \vspace*{\baselineskip}%
    \titleRule\\%[4\baselineskip]%
    \ifundef{\PROJECTTYPE}{}{\ifundef{\PROJECTNOTE}{\vspace*{4\baselineskip}}{\vspace*{3\baselineskip}}\usebox{\projecttypebox}\par}%
    \ifundef{\PROJECTNOTE}{}{\ifundef{\PROJECTTYPE}{\vspace*{4\baselineskip}}{\vspace*{3\baselineskip}}\usebox{\projectnotebox}\par}%
    \vspace*{\fill}%
    \usebox{\abstractbox}\par%
    % \vspace*{\fill}%
    % \includegraphics[width=40mm,alt={University of Edinburgh rondel}]{uoecrest.png}\par\SCHOOL\par%
    \vspace*{\fill}%
    \usebox{\staffbox}\\%
    \vspace*{\fill}%
    % \normalfont\titlepageFontSerif\Large\SCHOOL\par%
    \includegraphics[height=3\baselineskip,alt={School of Informatics, University of Edinburgh logo}]{\logofile}\par%
    \vspace*{\fill}
  \end{singlespace}%
  \thispagestyle{titlepage}%
\restoregeometry%
\clearpage
\pagenumbering{roman}%

\bookmarksetup{startatroot}%
\hypersetup{%
  ,linktoc=all%
  ,colorlinks=false%
  ,pdfborder={0 0 0}%
}%
\newcommand*{\NameTableContents}{Table of contents}% The content's display name
\renewcommand*\contentsname{\NameTableContents}%
\phantomsection\label{.toc}%
% If you DON'T want Table of contents itself appearing as an item in the table of contents then use this line
\hypertarget{.HTtoc}{}\bookmarksetup{startatroot}\bookmark[dest=.HTtoc]{\NameTableContents}%
% If you DO want Table of contents appearing in the table of contents then use this line
%\addcontentsline{toc}{section}{\NameTableContents}%

\pagestyle{contents}
\renewcommand{\cftsecfont}{\normalfont\sffamily\bfseries}%
\renewcommand{\cftsecpresnum}{\addfontfeatures{Numbers={Monospaced,Lining}}}%
\renewcommand{\cftsecaftersnumb}{\addfontfeatures{Numbers={Proportional,OldStyle}}}%
\renewcommand{\cftsecpagefont}{\normalfont\sffamily\bfseries\addfontfeatures{Numbers={Monospaced,OldStyle}}}%
\renewcommand{\cftsubsecfont}{\normalfont\sffamily}%
\renewcommand{\cftsubsecpresnum}{\addfontfeatures{Numbers={Monospaced,Lining}}}%
\renewcommand{\cftsubsecaftersnumb}{\addfontfeatures{Numbers={Proportional,OldStyle}}}%
\renewcommand{\cftsubsecpagefont}{\normalfont\sffamily\addfontfeatures{Numbers={Monospaced,OldStyle}}}%

\tableofcontents%
\newcommand*{\NameListFigures}{List of figures}
\phantomsection\label{.lof}%
\addcontentsline{toc}{section}{\NameListFigures}%
\renewcommand*\listfigurename{\NameListFigures}%
\listoffigures%
\phantomsection\label{.lot}%
\newcommand*{\NameListTables}{List of tables}
\addcontentsline{toc}{section}{\NameListTables}%
\renewcommand*\listtablename{\NameListTables}%
\listoftables%
\bookmarksetup{startatroot}%
\hypersetup{
   colorlinks=true
  ,linkcolor=hyperlinkColor
  ,linkbordercolor=hyperlinkColor% internal hyperlink border colour
  ,urlcolor=hyperlinkColor
  ,urlbordercolor=hyperlinkColor% external hyperlink border colour
  ,citecolor=hyperlinkColor
  ,citebordercolor=hyperlinkColor% internal citation border colour
}%

\clearpage
\pagenumbering{arabic}
\pagestyle{mainmatter}

%%%%%%%%%%
% You may start editing from here
\section{Introduction}%
\label{sec:introduction}

Introduce the topic of your research project and explain its academic, industrial, or societal context as appropriate.

\subsection{Motivation}%
\label{sec:motivation}

\begin{itemize}[nosep]
    \item What is the problem/topic you are going to work on.
    \item Why is this an important problem? Ideally include references/evidence.
    \item Provide an adequate introduction to the general area (fairly high level).
    \item If appropriate clearly state your hypothesis.
    \item Indicate the scope of your project and clearly identify your aims and objectives.
    \item Discuss the potential impact of your research including any possible beneficiaries.
    \item Remember to engage/enthuse your reader(s).
\end{itemize}

\subsection{Literature review}%
\label{sec:litreview}

\begin{itemize}[nosep]
    \item Include a short literature review. Demonstrate a knowledge and understanding of the most relevant past and current published research in the subject area. 
    \item Include relevant references like this \autocite{fry2020template} --- read \textcite{coxhead2016referencing} for how to use citations. You are writing using biber and biblatex NOT bibtex and natbib, so you MUST use the commands \verb|\autocite{}| and \verb|\textcite{}| NOT \verb|\cite{}|. The examples in this item show you how to use the two citation commands appropriately.
\end{itemize}



\section{Method or approach}%
\label{sec:method}

\begin{itemize}[nosep]
    \item In the context of your aims, objectives and literature review, detail the methodology or approach to be used in pursuit of the research and justify this choice. 
    \item Describe your planned contribution. Highlight what is new and where you will go beyond the state-of-the-art (eg new method(s), new tools,
    new data and/or analytical approach, new insights, new proofs,\ldots)
    \item Describe any specific methods need for data collection.
    \item Explain your evaluation; how do you intend to analyse and interpret your results.
\end{itemize}



\section{Work plan}%
\label{sec:workplan}

\begin{itemize}[nosep]
    \item Briefly explain the programme of work, indicating the research to be undertaken and a few milestones that can be used to measure your progress is on schedule.
    \item Define your major work packages and/or tasks and show any  dependencies between them. Workpackages, tasks and their dependencies should also be shown in a Gantt chart (example below).
    \item Explain how the project will be managed.
\end{itemize}

\subsection{Plan, milestones, risks, and deliverables}%
\label{sec:plan}

Include a Gantt chart and summary tables as required. A template is included below, but you may produce your own chart in another tool and include the image.

% Only use afterpage if there is a large gap between the paragraph above and the bottom of the page
\afterpage{
  % If you want to import a graphic of a Gantt chart use this code...

  % \begin{landscape} % turn the page sideways: displays correctly on screen and prints correcly
  %   \begin{figure}[!p]% alternatives: [!tbp] or [!btp] % DO NOT use [!h] or [h] or [H]
  %     \includegraphics[width=\lscapewidth,
  %     alt={A proper alt-text description of the Gantt contents is needed}]
  %     {filename}
  %     \caption[Project Gantt chart]{Gantt Chart of project activities, see \cref{tab:MDR} for details.}
  %     \label{fig:ganttImport}
  %   \end{figure}
  % \end{landscape}

% ...otherwise use this code to have LaTeX draw a Gantt chart for you
  \definecolor{canvasColor}{HTML}{FDFDFA}
  \definecolor{backgroundLineColor}{HTML}{B4B1A0}
  \definecolor{titlebarColor}{HTML}{E7E7DB}
  \definecolor{titlebarTextColor}{HTML}{5A5F55}
  \definecolor{groupColor}{HTML}{0C1B24}
  \definecolor{barColor}{HTML}{436382}
  \definecolor{barIncompleteColor}{HTML}{9F3C19}
  \definecolor{riskColor}{HTML}{9F3C19}
  \definecolor{linkColor}{HTML}{8AA0AD}
  \definecolor{milestoneColor}{HTML}{566038}

  \begin{landscape} % turn the page sideways: displays correctly on screen and prints correcly
    \begin{figure}[!p]% alternatives: [!tbp] or [!btp] % DO NOT use [!h] or [h] or [H]
      \hfill\begin{ganttchart}[
        expand chart=\lscapewidth, % use \textwidth for a vertical (portrait) page
        canvas/.append style={fill=canvasColor, draw=backgroundLineColor, line width=.75pt},
        y unit title=0.4cm,
        y unit chart=0.5cm,
        hgrid style/.style={draw=backgroundLineColor, line width=.75pt},
        vgrid={*{29}{dotted, draw=backgroundLineColor!50, line width=.75pt},
          *{1}{dashed, draw=backgroundLineColor!75, line width=.75pt},%30 days in June
          *{30}{dotted, draw=backgroundLineColor!50, line width=.75pt},
          *{1}{dashed, draw=backgroundLineColor!75, line width=.75pt},%31 in July
          *{30}{dotted, draw=backgroundLineColor!50, line width=.75pt},
          *{1}{dashed, draw=backgroundLineColor!75, line width=.75pt}%31 in August
        },
        x unit=1.4mm,
        time slot format=isodate,
        title/.append style={draw=none, fill=titlebarColor},
        title label font={\sffamily\bfseries\footnotesize\color{titlebarTextColor}},
        title left shift=.05,
        title right shift=-.05,
        title height=1,
        bar/.append style={draw=none, fill=barColor},
        bar height=.6,
        bar label font={\small\color{barColor}},
        bar incomplete/.append style={fill=barIncompleteColor},
        progress label text={},
        group label font={\sffamily\bfseries\small\color{groupColor}},
        group/.append style={draw=groupColor, fill=groupColor},
        group right shift=0,
        group top shift=.6,
        group height=.3,
        group peaks height=.2,
        link bulge=1,
        link/.style={->, linkColor},
        link type={auto},
        milestone label font={\bfseries\small\color{milestoneColor}},
        milestone/.append style={draw=none, fill=milestoneColor, scale=1.5},
        vrule/.append style={thin, riskColor}
        ]{2023-06-01}{2023-08-19}
        \gantttitlecalendar{month=name}\\
        \ganttgroup{Preparation}{2023-06-01}{2023-06-14}\\
        \ganttbar[progress=100,name=br]{Background work}{2023-06-01}{2023-06-14}\\
        \ganttbar[progress=100,name=dd]{Draft design}{2023-06-06}{2023-06-14}\\
        \ganttmilestone{M$_1$}{2023-06-14}\\
        \ganttgroup{Implementation}{2023-06-14}{2023-07-24}\\
        \ganttbar[progress=90, name=id]{Interface design}{2023-06-14}{2023-06-26}\\
        \ganttbar[progress=70, name=bs]{Back\=/end setup}{2023-06-14}{2023-06-26}\\
        \ganttmilestone{M$_2$}{2023-06-21}\\
        \ganttbar[progress=90, name=fd]{Feature development}{2023-06-26}{2023-07-15}\\
        \ganttmilestone{M$_3$}{2023-07-07}\\
        \ganttbar[progress=70, name=im]{Design improvement}{2023-07-10}{2023-07-24}\\
        \ganttbar[progress=100, name=co]{Compatibility checking}{2023-06-14}{2023-07-24}\\
        \ganttgroup{Results \& Evaluation}{2023-06-14}{2023-08-03}\\
        \ganttbar[progress=100, name=hci]{HCI evaluation}{2023-06-14}{2023-06-26}\\
        \ganttbar[progress=90, name=us]{First user study}{2023-07-10}{2023-07-15}\\
        \ganttmilestone{M$_4$}{2023-07-20}\\
        \ganttbar[progress=100, name=fus]{Final user study}{2023-07-24}{2023-07-29}\\
        \ganttbar[progress=100, name=ca]{Comparison \& analysis}{2023-07-29}{2023-08-03}\\
        \ganttmilestone{M$_5$}{2023-08-03}\\
        \ganttgroup{Report}{2023-06-01}{2023-08-19}\\
        % \ganttbar[progress=95, name=es]{$^{\text{*}}\,$Early sections}{2023-06-01}{2023-08-02}\\
        \ganttbar[progress=95, name=es]{Background \& experiments}{2023-06-01}{2023-08-02}\\
        \ganttbar[progress=100, name=er]{Evaluation}{2023-07-15}{2023-08-05}\\
        \ganttbar[progress=100, name=ia]{Introduction \& abstract}{2023-08-05}{2023-08-10}\\
        \ganttbar[progress=100,name=fw]{Finalising writing}{2023-08-10}{2023-08-19}\\
        \ganttbar[progress=100,name=pr]{Proofreading}{2023-06-01}{2023-08-19}\\
        \ganttmilestone{M$_6$}{2023-08-19}
        \ganttlink[link bulge=.75]{br}{dd}
        \ganttlink[link bulge=1.5, link mid=.7]{br}{id}
        \ganttlink[link bulge=2.5, link mid=.5]{br}{bs}
        \ganttlink{br}{fd}
        \ganttlink[link bulge=3.5, link mid=.12]{br}{hci}
        \ganttlink{bs}{fd}
        \ganttlink{ia}{fw}
        \ganttlink[link mid=.6]{im}{fus}
        \ganttlink{us}{fus}
        \ganttlink[link mid=.14]{ca}{er}
        \ganttlink[link mid=.1, link bulge=1.4]{fus}{er}
        \ganttlink[link mid=.16, link bulge=2]{us}{er}
        \ganttlink[link mid=.6]{us}{im}
        \ganttvrule{R$_{1\,\rightarrow}$}{2023-06-01}
        \ganttvrule{$_{\leftarrow\,}$R$_1$}{2023-06-14}
        \ganttvrule{R$_{2\,\rightarrow}$}{2023-07-02}
        \ganttvrule{$_{\leftarrow\,}$R$_2$}{2023-07-30}
        \ganttvrule{R$_{3\,\leftrightarrow}$}{2023-08-06}
        \ganttvrule{\,R$_3$}{2023-08-10}
        \ganttvrule{R$_{4\,\rightarrow}$}{2023-08-17}
      \end{ganttchart}
      \caption[Project Gantt chart]{Gantt Chart of project activities, see \cref{tab:MDR} for details. %
        % $^{\text{*}}$\,Early sections are the background, method, and experiments.
        Overarching bars coloured \textcolor{groupColor}{\rule{1ex}{\charHeight}}~represent the maximum time (including contingency) allocated per phase, \textcolor{barColor}{\rule{1ex}{\charHeight}}~represents the scheduled time for a task and \textcolor{barIncompleteColor}{\rule{1ex}{\charHeight}}~its contingency time. Arrowed lines indicate task dependencies}
      \label{fig:gantt}
      % This Gantt chart was devised by Trista Yang, MSc~2022, and extended by Brian Mitchell, 2023
    \end{figure}

    \begin{table*}[!p]% alternatives: [!tbp] or [!btp] % DO NOT use [!h] or [h] or [H]
      \begin{center}
          \begin{tabular}{lcS[table-format=2.0]*{2}{l}}
          \toprule
          & \textbf{Event} & \textbf{Week(s)} & \textbf{Description} & \textbf{Deliverable}\\
          \midrule
          & \textcolor{milestoneColor}{M$_1$} &
              2 & Feasibility study completed\\
          & \textcolor{milestoneColor}{M$_2$} &
              3 & First prototype implementation completed
              & Prototype\\
          & \bfseries\textcolor{milestoneColor}{M$_3$} &
              \bfseries 5 & \bfseries Second marker feedback\\
          & \textcolor{milestoneColor}{M$_4$} &
              7 & Implementation completed
              & Software repository\\
          & \textcolor{milestoneColor}{M$_5$} &
              7 & Evaluation completed
              & \bfseries Report draft\\
          \multirow{-6}{*}{\rotatebox{90}{\textcolor{milestoneColor}{Milestone}}}
          & \textbf{\textcolor{milestoneColor}{M$_6$}} &
              \bfseries 10 & \bfseries Submission deadline
              & \bfseries Final report\\
          \midrule
          & \textcolor{riskColor}{R$_1$} & 
              \text{1--2~\,} & First busy period for supervisor\\
          & \textcolor{riskColor}{R$_2$} & \text{4--8~\,} &
              Supervisor likely to be away\\
          & \bfseries\textcolor{riskColor}{R$_3$} & \bfseries 9 &
              \bfseries Changing accommodation\\
          \multirow{-4}{*}{\rotatebox{90}{\textcolor{riskColor}{Risk}}}
          & \textcolor{riskColor}{R$_4$} &  10 & Second busy period for supervisor\\
          \bottomrule
          \end{tabular} 
      \end{center}
      \caption[Project milestones, deliverables and risks]{Project milestones, deliverables, and risks (\textbf{bold} = main); see \cref{fig:gantt}}
      \label{tab:MDR}
    \end{table*}
  \end{landscape}
}% afterpage closing brace



\section{Responsible Research}
\begin{itemize}[nosep]
    \item You must include an ethical evaluation of your research --- even if your conclusion, which should be justified is that no ethical issues are relevant.
    \item If appropriate discuss any dual-use concerns
    \item Include a brief data management plan. How do you plan to package or handle any code, data or results for future use by yourself, your supervisor or anyone else.
    \item If appropriate discuss any licensing issues.
\end{itemize}

\printbibliography[heading=bibnumbered,title=\refname,label={bibliography}]

\appendix

\end{document}
